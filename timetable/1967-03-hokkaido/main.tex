\documentclass{article}
\usepackage{geometry}
\usepackage{hyperref}
\usepackage{ctex}

\setlength{\parindent}{0em}
\renewcommand{\abstractname}{Abstract}
\geometry{a4paper, scale=0.8}

\title{Hokkaido Train Timetable, 1 March 1967}
\author{\href{http://naeboworks.com/haisen/timetable.htm}{Naeboworks}; John Franklin}
\date{25 December 2024}

\begin{document}
	\sffamily
	\maketitle 
	\begin{abstract}
		This document records the timetable of passenger trains in Hokkaido, Japan's northernmost main island, on 1 March 1967 (Showa 42), which was during the peak of Hokkaido Railway System, with many trains and railway lines at that time being long abandoned currently. Source of content: \href{http://naeboworks.com/haisen/timetable.htm}{Naeboworks (苗穂工房)}.
	\end{abstract}
	\par 
	Signs and marks are as follows: \\
	\textbf{Stations: }\\
	S: Stations with a shop selling soba noodles; \\
	B: Stations selling bentos (Japanese packed fast food); \\
	D: Stations with ability to handle telegraphs; \\
	\textbf{Train names: }\\
	J: Semi-Express (Junkyu, 準急); \\
	K: Express (Kyukou, 急行); \\
	T: Limited Express (Tokkyu, 特急); \\
	\textbf{Train Seats: }\\
	Z: Second-class seats, reservation needed; \\
	Y: First-class seats, reservation needed; \\
	W: Reservation needed for any seat of the whole train; \\
	B: B-class sleeper; \\
	A: A-class sleeper; \\
	S: General sleeper (Shindai, 寝台); \\
	R: Restaurant car; \\
\end{document}